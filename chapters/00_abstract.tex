\begin{zusammenfassung}

    In diesem Forschungsprojekt wurden nützliche topologische Indizes für bestimmte Klassen und Strukturen von Graphen im Bereich der Graphentheorie untersucht.
    Durch die Berechnung diverser Indizes für eine Reihe von Graphen und den Vergleich ihrer Leistung wurde in der Studie ermittelt, welche Indizes für verschiedene Graphenklassen am nützlichsten sind. \\
    Auf der Grundlage dieser Erkenntnisse wurde eine Anwendung entwickelt, die anhand eines Eingabegraphen den einflussreichsten topologischen Index für die jeweilige Klasse bestimmt.
    Diese Anwendung hat das Potenzial, ein wertvolles Werkzeug für Forscher und Praktiker zu sein, die mit Graphdaten arbeiten, und könnte für ein breites Spektrum eingesetzt werden, darunter molekulare, bioinformatische, soziale und synthetische Graphen. \\
    Insgesamt bieten die Ergebnisse dieser Studie wesentliche Einblicke in die Verwendung von topologischen Indizes in der Graphenanalyse und können Forschung und Praxis in diesem Bereich potenziell beeinflussen.

\end{zusammenfassung}

\keywords{Graphentheorie, Topologische Indizes, Statistische Analyse, Graphenklassifikation}

\newpage

\begin{abstract}

    This thesis presents the findings of a comprehensive study on the usefulness of topological indices for specific classes and structures of graphs in graph theory.
    Using a variety of graphs, including molecular, bioinformatics, social, and synthetic graphs, this study calculated a range of topological indices and compared their performance on each graph class.  \\
    The results indicated that different indices had varying degrees of influence on the principal components of graph classes, with some indices being more useful than others for particular classes of graph. \\
    Based on these findings, the study developed an application that classifies an input graph into one of the four graph classes. The application then returns the most influential topological index for the given class, as determined by the results of the principal component analysis. The findings of this study have significant implications for the use of topological indices in a wide range of applications and provide valuable insights for researchers and practitioners working with graph data.

\end{abstract}

\keywords{Graph theory, Comparative analysis, Topological indices, Principal component analysis, Graph classification}
