\section{Applikation und Code}

Es folgt der Code zur Entwicklung eines Python-Moduls, welches die Berechnung der nützlichen topologischen Indizes in verschiedenen Graphenklassen ermöglicht.
Das Modul kann auf GitLab unter \url{https://git.ffhs.ch/luca.hostettler/bt-hostettler} gefunden werden.

Es ist in zwei Schritte unterteilt:
\begin{enumerate}[itemsep=0pt]
    \item \textbf{Eingabe eines Graphen,}
    \item \textbf{Empfehlungen der topologischen Messwerte}
\end{enumerate}

\subsection{Eingabe des Graphen}

Das Programm akzeptiert als Eingabe einen NetworkX-Graphen.
Aus der Datenanalyse wurde die Usefulness einzelner topologischer Indizes für verschiedene Netzwerkarten ermittelt. Dabei wurden die Netzwerke in folgende Kategorien eingeteilt: \textbf{Random}, \textbf{Small-World}, \textbf{Scale-free}, \textbf{Complete}, \textbf{Line}, \textbf{Tree} und \textbf{Star}. Die Klassifizierung des Graphen erfolgt mit dem GCN-Modell aus Abschnitt \ref{sec:classification}, welches bereits beschrieben und trainiert wurde.

\begin{listing}[H]
    \begin{minted}[frame=lines,framesep=2mm,baselinestretch=1.2,bgcolor=LightGray,fontsize=\scriptsize,linenos]{python}    
# create an instance of the app
app = App()
# create a sample graph using networkx
graph = nx.path_graph(150)
# test graph classification
g_class = app.classify(graph)

print(f"Found graph class {app.class_keys[g_class]}")
# get the topological indices
topological_indices = app.get_topological_indices(graph)
# print the topological indices
print(topological_indices)
    \end{minted}
    \caption{Code zum Einlesen des Graphen}
\end{listing}

Wie im vorhergehenden Code-Listing ersichtlich, ist der benötigte Code für die Applikation überaus kurz und einfach. Dabei ist \mintinline{python3}{App} die Klasse, welche die Applikation implementiert. Die Methode \mintinline{python3}{classify} ist für die Klassifikation des Graphen zuständig. Die Methode \mintinline{python3}{get_topological_indices} ist für das Laden und Anzeigen der für den Graphen nützlichen topologischen Indizes verantwortlich.

\subsection{Das GCN-Modell}

Das Rezept für das Modell besteht aus in drei Schritten:

\begin{enumerate}
    \item jeden Knoten mit Node-Embedding verarbeiten,
    \item alle Node-Embeddings in ein gemeinsames Graph-Embedding zusammenfassen (ReadOut-Layer) und
    \item die Klassifizierung auf Graph-Embedding trainieren.
\end{enumerate}

\begin{listing}[H]
    \begin{minted}
        [frame=lines,framesep=2mm,baselinestretch=1.2,bgcolor=LightGray,fontsize=\footnotesize,linenos]
        {python}    
class GCN(torch.nn.Module):
    def __init__(self, hidden_channels=64, num_node_features=3, num_classes=7):
        super(GCN, self).__init__()
        torch.manual_seed(12345)
        self.conv1 = GCNConv(num_node_features, hidden_channels)
        self.conv2 = GCNConv(hidden_channels, hidden_channels)
        self.conv3 = GCNConv(hidden_channels, hidden_channels)
        self.lin = Linear(hidden_channels, num_classes)

    def forward(self, x, edge_index, batch):
        # 1. Obtain node embeddings
        x = self.conv1(x, edge_index)
        x = x.relu()
        x = self.conv2(x, edge_index)
        x = x.relu()
        x = self.conv3(x, edge_index)

        # 2. Readout layer
        x = global_mean_pool(x, batch)  # [batch_size, hidden_channels]

        # 3. Apply a final classifier
        x = F.dropout(x, p=0.5, training=self.training)
        x = self.lin(x)

        return x
    \end{minted}
    \caption{Der Code für das GCN-Modell}
    \label{lst:code_gcn}
\end{listing}

\subsection{Code für die Klassifikation des Graphen}

Das oben aufgeführte Modell \ref{lst:code_gcn} wird verwendet, um die Klassifikation des Graphen zu bestimmen.
Diese wird in zwei Schritten durchgeführt:

\begin{enumerate}
    \item \textbf{Aufbereitung der Eingabe}
    \item \textbf{Klassifikation des Graphen}
\end{enumerate}

\begin{listing}[H]
    \begin{minted}
        [frame=lines,framesep=2mm,baselinestretch=1.2,bgcolor=LightGray,fontsize=\footnotesize,linenos]
        {python}    
# classify the graph
def classify(self, G):
    node_labels = np.arange(G.number_of_nodes())
    degrees = nx.degree_centrality(G)
    betweenness = nx.betweenness_centrality(G)
    attrs = dict(zip(node_labels, node_labels))
    nx.set_node_attributes(G, attrs, "label")
    nx.set_node_attributes(G, degrees, "degree")
    nx.set_node_attributes(G, betweenness, "betweenness")
    # convert to torch tensor
    x = from_networkx(graph, group_node_attrs=["label", "betweenness", "degree"])
    test_loader = DataLoader([x], batch_size=1, shuffle=False)
    # run the model
    for x in test_loader:
        out = self.model(x, x.edge_index, x.batch)
        # convert to numpy
        pred = out.argmax(dim=1)  # Use the class with highest probability
        # return the result
        return pred
    \end{minted}
    \caption{Code zur Klassifikation des Graphen}
\end{listing}

\subsection{Empfehlungen der topologischen Messwerte}

In der Konsole werden, zusammen mit der vorhergesagten Klasse des Graphen, die vorgeschlagenen topologischen Indizes ausgegeben.

\begin{listing}[H]
    \begin{minted}
        [frame=lines,framesep=2mm,baselinestretch=1.2,bgcolor=LightGray,fontsize=\footnotesize,linenos]
        {python}    
# get the topological indices
def get_topological_indices(self, graph):
    # classify the graph
    y = self.classify(graph)
    # get the topological indices
    topological_indices = y[0]
    # return the topological indices
    return topological_indices
    \end{minted}
    \caption{Code für den Vorschlag der topologischen Indizes}
\end{listing}

In einem späteren Schritt wird zusammen mit den vorgeschlagenen topologischen Indizes eine Erklärung geliefert.

Die Idee ist, dass die Erklärung zur \textit{Explainability} beiträgt. 
Als Kontext sind zwei Punkte relevant: zum einen die topologischen Indizes, welche vorgeschlagen werden, und zum anderen die vorhergesagte Klasse des Graphen.