\section{Die Entwicklungsumgebung}

\subsection{Gründe für den Einsatz von Python}

Die Programmiersprache Python wird in dieser Recherche für die Netzwerkanalyse eingesetzt, obwohl auch die Programmiersprache R eine Vielzahl an Paketen für diese Aufgabe bereitstellt.

Allerdings ist Python auf GitHub weiterverbreitet als R.
Ein wesentlicher Vorteil von Python im Vergleich zu R, besteht in der Nutzung von Jupyter-Notebooks, welche die Entwicklung und Dokumentation der Netzwerkanalyse erleichtern.
Durch die Kombination von Experimenten, Daten und Ergebnissen in einem Dokument können Letztere effektiv präsentiert werden.
Matplotlib wird verwendet, um die Ergebnisse in Diagrammen darzustellen.

\subsection{NetworkX}

NetworkX ist die bekannteste und am meisten verwendete Bibliothek für die Netzwerkanalyse in Python \cite{hagberg_exploring_2008}.
Die Anwendung ist simpel und die Dokumentation weit fortgeschritten.

Ein Beispiel für das Erstellen eines einfachen Graphen mit NetworkX ist:

\begin{listing}[H]
    \begin{minted}[frame=lines,framesep=2mm,baselinestretch=1.2,bgcolor=LightGray,fontsize=\footnotesize,linenos]{python3}
import networkx as nx

G = nx.Graph()
G.add_node(1)
G.add_nodes_from([2, 3])
G.add_edge(1, 2)
G.add_edges_from([(1, 2), (1, 3)])
    \end{minted}
    \caption{Erstellen eines einfachen Graphen mit NetworkX}
\end{listing}

\subsection{GrinPy}

GrinPy ist eine Bibliothek für die Netzwerkanalyse in Python \cite{amos_grinpy_2022}, die auf die Analyse von Graphen mit Attributen spezialisiert ist.
Die Dokumentation ist noch nicht weit fortgeschritten.

GrinPy ist eine Erweiterung von \mintinline{python3}{NetworkX} und besitzt eine Vielzahl an Funktionen, um topologische Indizes von Graphen zu berechnen.
Somit können topologische Indizes von Graphen, welche durch NetworkX erstellt wurden, direkt berechnet werden. Die Graphen müssen nicht erst in ein anderes Format konvertiert werden.

\subsection{Matplotlib}

Matplotlib ist eine Bibliothek für die Visualisierung von Daten in Python \cite{Hunter:2007} und wird eingesetzt, um die Ergebnisse der Netzwerkanalyse in Diagrammen darzustellen.
Dies ist für die explorative Datenanalyse und das Verständnis der Ergebnisse nützlich.

Auch kann die Bibliothek verwendet werden, um die Ergebnisse in einem Dokument zu visualisieren.
Die Integration in die Skripts und Jupyter-Notebooks ist unkompliziert. Ergebnisse können direkt ausgegeben und betrachtet werden.
Zudem können die Bilder auch in verschiedenen Formaten zur weiteren Bearbeitung abgespeichert werden.

\subsection{PyTorch Geometric}

PyTorch Geometric \cite{fey_lenssen_2019} ist eine Graph-Machine-Learning-Bibliothek, welche eine Erweiterung von PyTorch \cite{Paszke_PyTorch_An_Imperative_2019} entwickelt darstellt.
Sie ist entwickelt worden, um schnell und einfach Graph Neural Networks zu implementieren.
Sie kann für verschiedene Anwendungen eingesetzt werden, unter anderem für \textbf{Node-Level-Predictions}, \textbf{Edge-Level-Predictions} oder in diesem Fall \textbf{Graph-Level-Predictions}.

Die Bibliothek ist optimal dokumentiert und bietet eine Vielzahl an Beispielen, welche die Implementierung von Graph Neural Networks erleichtern.

\subsection{Pandas}

Pandas ist eine Bibliothek für die Datenanalyse in Python \cite{mckinney-proc-scipy-2010}.
Im Kontext der Arbeit eignet sich Pandas besonders, um die berechneten topologischen Indizes in einem Data-Frame zu speichern.
Wie bereits in Code-Listing \ref{lst:indices} beschrieben, wird als Index der Tabelle jeweils der Name (Index) des Graphen verwendet.
Die Spalten des Data-Frames sind die verschiedenen Werte der topologischen Indizes.

Diverse statistische Methoden sind direkt in Pandas implementiert. 
Unter anderem können die Mittelwerte, Standardabweichungen und Mediane direkt berechnet werden. 
Aber auch die Korrelation zwischen den jeweiligen topologischen Indizes kann vereinfacht angezeigt und ausgegeben werden.
